\documentclass[preprint2]{aastex62}

\bibliographystyle{aasjournal}
\usepackage{graphicx}
\usepackage[suffix=]{epstopdf}
\usepackage{natbib}
\usepackage{amsmath}
\usepackage{url}
\usepackage{xspace}

%    Make Scientific Notation
\providecommand{\e}[1]{\ensuremath{\times 10^{#1}}}

% make the word Kepler italicized
\newcommand{\Kepler}{\textsl{Kepler}\xspace}



\begin{document}
%%%%%%%%%%%%%%%%%%%%%%
\title{The Evolution of Flare Activity with Stellar Age from \Kepler}

\shorttitle{Flares versus Age}
\shortauthors{Davenport et al.}


\correspondingauthor{James. R. A. Davenport}
\email{James.Davenport@wwu.edu}

\author{James. R. A. Davenport}
\altaffiliation{NSF Astronomy and Astrophysics Postdoctoral Fellow}
\altaffiliation{DIRAC Fellow}
\affiliation{Department of Physics \& Astronomy, Western Washington University, 516 High St., Bellingham, WA 98225, USA}
\affiliation{Department of Astronomy, University of Washington, Seattle, WA 98195, USA}

\author{Kevin R. Covey}
\affiliation{Department of Physics \& Astronomy, Western Washington University, 516 High St., Bellingham, WA 98225, USA}

\author{Riley W. Clarke}
\affiliation{Department of Physics \& Astronomy, Western Washington University, 516 High St., Bellingham, WA 98225, USA}

\author{Austin C. Boeck}
\affiliation{Department of Physics \& Astronomy, Western Washington University, 516 High St., Bellingham, WA 98225, USA}

\author{Jonathan Cornet}
\affiliation{Department of Physics \& Astronomy, Western Washington University, 516 High St., Bellingham, WA 98225, USA}

\author{Suzanne L. Hawley}
\affiliation{Department of Astronomy, University of Washington, Seattle, WA 98195, USA}

 

%%%%%%%%%%%%%%%%%%%%%%%%%%%%%%
\begin{abstract}
351 stars
results from our automated survey of stellar flares using the entire Kepler dataset. Matching our flare stars to Kepler rotation periods, we find a decline in the energy emitted in flares as stars spin down. For a subset of Kepler M dwarfs with low resolution followup spectra, we also find a correlation between Halpha luminosity and the energy emitted in flares. These two results give the first definitive evidence of flare rates declining over stellar time, indicating flares are intimately connected to age--rotation--activity evolution of the global stellar dynamo. 
\end{abstract}


%%%%%%%%%%%%%%%%%%%%%%%%%%%%%%
\section{Introduction}

Surface magnetic activity is observed to decline over time for low-mass stars on the main sequence. Magnetic activity comes in a wide range of observable phenomena, including UV and X-ray luminosity, cool starspots, and flares. 
As the star loses angular momentum via stellar winds, the rotation velocity decreases and the internal magnetic dynamo is quieted. This age--activity connection was outlined in the seminal work by \citet{skumanich1972}, which was directly connected with flare activity as well \citep{skumanich1986}. Magnetic activity evolution has been confirmed and updated for low-mass stars (late F through early M type) in many studies using X-ray luminosity \citep{wright2011}, UV emission \citep{shkolnik2014}, Zeeman-Doppler imaging \citep{vidotto2014}, and H$\alpha$ emission \cite{west2015}, to name but a few. 


This surface activity can have significant impact on the evolution of habitable zone planets orbiting active stars. Flares in particular have been studied as a possible threat to a planet's ability to retain a habitable atmosphere \citep[e.g.][]{segura2010,luger2015,tilley2017}. For example, UV flux and high energy particles impacting a terrestrial planet's atmosphere from frequent stellar flares can significantly deplete ozone, and result in a potentially uninhabitable planetary surface \citep{tilley2017}. Under more extreme scenarios, stellar activity could strip large portions of a planet's atmosphere away over timescales of a few 100 Myr \citep{luger2015}. The duration of high flare activity early in a star's life may therefore be a fundamental property in defining potential planetary habitability. 


Stellar activity also makes planet searches more difficult, adding both slow variability (e.g. starspots) that can impact radial velocity studies, and fast stochastic variability (e.g. flares) that can impede transit searches \citep[e.g.][]{kipping2017}. Further, though stellar activity is less prominent at redder wavelengths, even observing in the infrared does not eliminate the potential for flares from active stars to impact transit searches, particularly for the lowest mass stars \citep{davenport2017a}. To understand planet occurrence rates for nearby low-mass stars, as well as the evolution of planetary habitability, we must constrain flare rates and properties as a function of stellar age.


Measuring the evolution of flare activity with stellar age has typically been limited to comparisons between flare stars in young clusters or stellar associations with known ages. Studying flare stars in young nearby moving groups and clusters reaches back at least five decades \citep{haro1966}, with the conclusion that all low-mass dwarf stars undergo an evolution in flare activity \citep{ambartsumian1975}. Pre-main sequence stars have also been known to exhibit high levels of flare activity compared to field-aged dwarfs \citep[e.g.][]{feigelson2001}. However, in most previous studies field stars are assumed to be approximately Solar-aged, and therefore generally flare-inactive, since few reliable age indicators for isolated field stars exist.


Space-based exoplanet transit missions such as \Kepler \citep{borucki2010} have provided a revolutionary dataset for statistical studies of flare stars, particularly for field dwarfs \citep{walkowicz2011}. Such missions are ideally suited for large scale studies of flares, as they produce high precision, continuous light curves for months to years in duration.
The \Kepler survey has been used to explore ``super-flare'' activity (events with energies more than 10 times larger than those observed on the Sun) from solar-mass stars \citep{shibayama2013}, as well as from K and M dwarfs \citep{candelaresi2014}. White-light flares in \Kepler light curves have been detected across the main sequence, from massive A and F stars \citep{balona2012} down to L dwarfs \citep{gizis2013}, and produced the most detailed catalogs of flares for individual active stars to date \citep{hawley2014,davenport2014b}. 


In this paper we present an ensemble analysis of flare activity in the \Kepler field, based on the flare sample amassed in \citet{davenport2016}. This sample was generated using an automated processing of the entire \Kepler light curve database, and produced a sample of over 4000 candidate flare stars. The \citet{davenport2016} catalog of flares provides the most complete census of flare activity from a large sample of field stars to date, and is the ideal dataset to study the evolution of flare rates with stellar age.


Here we explore the relationship between flare occurrence rates and stellar ages (derived from rotation periods) for stars in the \Kepler field. Our sample of flare stars with measured rotation periods is detailed in \S\ref{sec:sample}.
We present three approaches for quantitatively tracing changes in flare activity over time, including modeling the flare frequency distribution as a function of both stellar mass and age in \S\ref{sec:activity}. In \S\ref{sec:model} we explore the empirical evolution of flare activity for our \Kepler sample, and provide an analytic prescription for use in other studies. Finally, a short discussion and comparison to other studies is given in \S\ref{sec:discussion}.




%%%%%%%%%%%%%%%%%%%%%%%%%%%%%%
\section{Flare Sample}
\label{sec:sample}

Our sample of flare stars comes from the automated search of \Kepler light curves from \citet{davenport2016}. This study produced the most comprehensive analysis of the \Kepler field for stellar flares, processing every short (1-minute) and long (30-minute) cadence light curve in search of flares. \citet{davenport2016} produced an open-source Python flare analysis codebase named {\tt appaloosa}, designed to detrend (model) \Kepler light curves of both instrumental and astrophysical noise, detect flare candidates (positive, significant outliers), and determine the reliability of the detected flares via artificial flare injection and recovery tests. These completeness tests were run on each continuous local segment of light curve available for every star, resulting in variable completeness limits for each star as the light curve noise properties change. \citet{davenport2016} made both the {\tt appaloosa} code, and the code to generate the figures and results in the paper available online. The results presented in this paper directly extend the work of \citet{davenport2016}, and so we also make our analysis code available as an update to the {\tt appaloosa} project online.\footnote{\url{https://github.com/jradavenport/appaloosa}}


The \citet{davenport2016} sample of potential \Kepler flare stars required that within all available data for each star at least 100 candidate flares of any energy, and 10 flares with energies above the 68\% completeness threshold determined by the automated flare injection tests, be recovered. These conservative thresholds potentially could eliminate stars with fewer significant flares from our analysis, as noted by \citet{van-doorsselaere2017}. In \S\ref{sec:activity} and \S\ref{sec:model} we use both a sample of flare stars that pass these cut established by \citet{davenport2016}, and a sample that have no flare event thresholds. The latter is important for exploring the evolution of flare rates down to very low activity states, and indeed for considering null detections appropriately. Note also that stars with no flare event detections by {\tt appaloosa} will still have upper-limits on their flare detectability from the artificial injection tests.



Since we are focused here on quantifying the changes of flare activity with stellar age, we limit our analysis to stars with measured rotation periods. This is so we can use stellar rotation period as a proxy for age. We note that while the rotation--age connection may be more complex than most gyrochronology prescriptions, e.g. the weakened rotational braking found in Solar-age stars from \citet{van-saders2016}, rotation nonetheless increases continuously with age (i.e. stars nominally do not speed up while on the main sequence). Rotation is therefore a good means to {\it sort} stars by their age, even if the specific age derived from gyrochronology relations is not accurate. Our work here is further insulated from these effects, as the sample of flare star candidates from \citet{davenport2016} predominantly have rotation ages of $\lesssim 1$ Gyr. We specifically adopt for our analysis the rotation period catalog of \citet{mcquillan2014}, which derived 34,040 periods from \Kepler data using the Autocorrelation Function, and did extensive testing against other period-finding approaches such as the Lomb-Scargle Periodogram. 


The \citet{davenport2016} sample of $\sim$4,000 candidate flare stars contains several types of contamination from variable stars that do not exhibit flaring activity. For example, we found some eclipsing binaries and pulsating stars were able to fool the {\tt appaloosa} code, due to sharp features in their light curve, particularly at the 30-minute cadence. The worst cases of this failure appear to be caused by insufficient modeling of periodic signals by {\tt appaloosa}, namely by using too few sine curves to fit each significant period found and leaving sharp or peaky structures in the model residuals. Future versions of {\tt appaloosa} will improve on this detrending algorithm, and we urge caution when adopting the \citet{davenport2016} flare sample blindly. Our analysis is largely free from such contamination, as we require each star to have an identified rotation period by \citet{mcquillan2014}, who rejected such eclipsing and pulsating targets.



In Figure \ref{fig:ffd1} we show light curves and cumulative flare frequency distributions (FFDs) for three example flare stars from the \citet{davenport2016} sample that also have rotation periods measured in \citet{mcquillan2014}. These example stars will be followed through the analysis of the paper, and were selected to demonstrate a range of flare rates and rotation periods. In total we study the flare rate evolution from 3XX stars with rotation periods from \citet{mcquillan2014} that pass the \citet{davenport2016} flare selection criteria. We also analyze 2Y,YYY stars with rotation periods that were searched for flares, some with marginal levels of flare activity.

%all stars from Amy's sample included. this gives us NNN stars with good flare detections from D16 in the McQ sample, and YYY stars from that sample that were searched for flares but not found significantly flaring. 


\begin{figure*}[!t]
\centering
\includegraphics[width=2.25in]{figures/6117832lc.png}
\includegraphics[width=2.25in]{figures/7420545lc.png}
\includegraphics[width=2.25in]{figures/10071383lc.png}\\
\includegraphics[width=2.25in]{figures/6117832_ffd}
\includegraphics[width=2.25in]{figures/7420545_ffd}
\includegraphics[width=2.25in]{figures/10071383_ffd}
\caption{
Three examples of flare stars from the \citet{davenport2016} sample. top row: sample light curves. Bottom row: cumulative flare frequency distributions (FFDs) from {\tt appaloosa} flare finding from \citet{davenport2016}. Each FFD is fit with a power law
}
\label{fig:ffd1}
\end{figure*}





%%%%%%%%%%%%%%%%%%%%%%%%%%%%%%
\section{Quantifying Flare Activity}
\label{sec:activity}

In order to accurately measure the evolution of flare rates over time, we must find a suitable metric to characterize flare activity for an ensemble of stars. Typical magnetic activity indicators, such as H$\alpha$ or X-ray flux, are used as disk-integrated measures of magnetically-driven emission from the chromosphere or corona. These quantities are are often presented as relative luminosities, normalized either to a continuum flux, or to the stellar bolometric luminosity.  While the overall rate and maximum intensity of flares for a given star are related to the global magnetic field strength, the specific properties of individual flare events  (e.g. duration, amplitude, morphology) are dependent on small-scale magnetic active regions. Since flare energy is inversely proportional to event occurrence frequency, the measured flare properties for a given star are also dependent on observing depth and baseline. Our flare activity metric must therefore represent the integrated properties of many individual flare events to accurately model a star's magnetic activity state.


In this section we outline three methods for quantifying the photometric flare activity between stars, as well as the merits and challenges of each approach. These metrics include: 1) the fractional energy emitted in the \Kepler band by flares ($L_{fl}/L_{Kp}$), 2) the cumulative flare rate evaluated at a specific energy, and 3) an analytic model of the entire flare frequency distribution (FFD). While all three of these metrics have utility, we believe the latter will be of most value to future investigations, for example in studies of planetary habitability and atmosphere evolution. 

We note that the pedagogical discussion here of comparing flare metrics between \Kepler stars is similar in many ways to the review on flare activity by \citet{kunkel1975}. This excellent review explored two different methods for quantifying and comparing flare activity from heterogeneous photometric studies: the integral of the light curve, and the rate at a given specific energy level, which are directly analogous to the first two approaches advanced here. The third approach, comparing the entire FFD between stars, has a long history \citep[e.g. see Fig. 17 of][]{lme1976}, and has even been used for comparing \Kepler flare stars \citep[e.g.][]{ramsay2013,hawley2014}.




%%%%%%%%%%%%%%%%%%%%%%%%%%%%%%
\subsection{Fractional Flare Luminosity}
\label{sec:fracL}

An intuitive metric for quantifying magnetic activity strength via flares is the total luminosity emitted by flares relative to the nominal quiescent stellar luminosity, written originally as $L_{fl}/L_{Kp}$ by \citet{lurie2015}. This quantity is inspired by traditional magnetic activity measures for low-mass stars, such as $L_{H\alpha}/L_{bol}$  \citep{walkowicz2004} or $L_X/L_{bol}$ \citep{pallavicini1981}, which are normalized relative to the bolometric stellar luminosity. In this case, as presented in \citet{lurie2015}, the flare luminosity is normalized to the quiescent luminosity of the star {\it only} in the \Kepler bandpass. This metric was used by \citet{davenport2016}, and also recently by \citet{yang2017}. % to measure decline in flare activity versus rotation and Rossby number

{\bf consider: the PSI factor}

The relative flare luminosity has many advantages as a magnetic activity strength indicator. First, it is algorithmically simple to compute by taking the integral of the flaring portion of the light curve, as described by \citet{kunkel1975}. For \Kepler light curves, this is done by de-trending the non-flaring (quiescent) light curves, including starspot variations, normalizing them to their average relative flux, and then integrating all the identified flares. Integrating the relative flux of a single flare results in a quantity known as the ``equivalent duration'' \citep[e.g. see][]{huntwalker2012}, which has units of time (typically seconds). By integrating the relative flux from {\it all} flares in a \Kepler light curve, we would again have units of time, and so as defined by \citet{lurie2015} $L_{fl}/L_{Kp}$ is simply computed as the integral of the relative flux of all flares, divided by the total observation duration of the light curve, resulting in a unit-less ratio.



A second appealing aspect of $L_{fl}/L_{Kp}$ as a flare activity indicator is that it compresses the entire observed flare activity of a star, regardless of the duration of the observation window, into a single number. This results in a quantity that has higher signal-to-noise than a specific flare rate, for example. This makes $L_{fl}/L_{Kp}$ ideal for comparing flare activity between stars, even with different observing baselines (e.g. comparing flare activity between active stars with differing numbers of quarters observed by \Kepler).

Thirdly, the light curve data does not need to be flux calibrated, or have accurate distances determined to measure $L_{fl}/L_{Kp}$. Instead the metric is defined totally by the relative flux increases of the flares. This is especially useful for datasets like \Kepler, where the light curves have incredible short-term precision designed to detect small amplitude exoplanet transits, but suffer from large-scale systematics that typically prevent flux calibration. As in the exoplanet transit application, the \% change of the light curve is the only quantity required. This also results in $L_{fl}/L_{Kp}$ being easily compared for many stars at once. \citet{lurie2015}, for example, used $L_{fl}/L_{Kp}$ to measure flare activity for the M5+M5 binary system GJ 1245 AB, and to relate this flare activity to other mid-to-late type M dwarfs such as GJ 1243 \citep{davenport2014b}. 



% made in a standalone ipython notebook
\begin{figure}[!t]
\centering
\includegraphics[width=3.5in]{figures/flare_psi}
\caption{
an approximation of the flare $\chi$, the correction factor that can be used to convert the observed fractional flare energy  $L_{fl}/L_{Kp}$ in to the metric for comparing between stars of different masses, $L_{fl}/L_{bol}$. This $\Psi$ was computed using the bolometric and \Kepler absolute magnitudes from a 600Myr PARSEC isochrone \citep{bressan2012}. Here we show this correction factor versus $g-i$ color as a proxy for mass or spectral type \citep[e.g. see][]{covey2007,davenport2014}.
}
\label{fig:chi}
\end{figure}


As \citet{lurie2015} note, to correctly compare flare activity between stars of varyings spectral types (or effective temperatures), the measured quantity $L_{fl}/L_{Kp}$ requires a bolometric luminosity correction. Specifically, a correction must be made for the varying portion of the bolometric flux that is observed within the \Kepler bandpass. This is akin to the ``$\chi$'' parameter, first developed to convert H$\alpha$ equivalent width measurements into  $L_{H\alpha}/L_{bol}$, developed by \citet{walkowicz2004}. \citet{douglas2014} recently produced a thorough discussion on developing a $\chi$ factor using model spectra, and produced an updated table of $\chi$ values as a function of photometric colors in many bands. We chose to use the letter $\Psi$ as it follows $\chi$ in the Greek alphabet.

In Figure \ref{fig:chi} we demonstrate a similar parameter that can be used for converting measured $L_{fl}/L_{Kp}$ values into $L_{fl}/L_{bol}$, and thus more accurately compare the flare activity level between stars of different masses. The parameter $\Psi$ was determined using the \Kepler and bolometric luminosities computed for main sequence stars in a 600 Myr isochrone from the PARSEC model grid \citet{bressan2012}. The very wide bandpass of the \Kepler filter means the $\Psi$ factor is relatively close to 1 for most stars, and doesn't change much between spectral types. For ease of use, we also provide a simple polynomial fit to the curve shown in Figure \ref{fig:chi}:
\begin{eqnarray}
\log \Psi =& -0.0013 (g-i)^4 + 0.021 (g-i)^3 \notag \\
& -0.146 (g-i)^2 + 0.105 (g-i) \notag \\ 
& + 0.004
\end{eqnarray}


Note that the $\Psi$ parameter assumes a ``gray'' spectral response for the flare itself over the \Kepler bandpass. While H$\alpha$ emission occurs over a fairly small range of wavelength for most stars, flares emit energy over all observed wavelengths. The shape of this emission in optical wavelengths is typically characterized as a hot blackbody, usually with $T_{eff}\approx10,000$ K, but has been seen to have significant excess emission in both the blue and red \citep[e.g. see][]{kowalski2013}. This effective temperature also changes throughout the flare event in a manner that is not well characterized, particularly for complex, multi-peaked flare events \citep[e.g.][]{slhadleo,kowalski2012}. \citet{davenport2016} discussed this in terms of estimating the energy of a single flare event, assuming a gray flare response across the \Kepler band. Some other studies assume a flare spectral model, which can in turn imply higher energies for specific events \citep[e.g.][]{gizis2013,maehara2015}. The $\Psi$ parameter shown in Figure \ref{fig:chi} will similarly underestimate the flare luminosity. However, as in \citet{davenport2016} we believe the best approach is to assume a gray response and not assume a single flare spectrum for all moments of every flare event.

%thus the contrast between flare and quiescent emission is dependent on both the stellar spectrum and the flare temperature profile. Kowalski citation for detailed work on building a real flare SED in the optical, which might be promising way forward. 

\begin{figure}[!t]
\centering
\includegraphics[width=3.5in]{figures/Rossby_lfllkp_color}
\includegraphics[width=3.5in]{figures/Rossby_lfllbol_color}
\caption{
Rossby number vs flare energy for sample from Paper 1.
Top: The original $L_{fl}/L_{Kp}$ metric, as shown in \citet{davenport2016} versus Rossby number, with points colored by the stellar masses estimated by \citet{davenport2016} .
Bottom: The new $L_{fl}/L_{bol}$ metric, corrected using the $\Psi$ parameter, with point colors as above.
}
\label{fig:rossby1}
\end{figure}
  

\citet{davenport2016} found that the fractional flare luminosity, $L_{fl}/L_{Kp}$, decreased with increasing Rossby number, which is defined as the rotation period divided by the convective turnover timescale (Ro = P$_{rot} / \tau$). This result indicates that the flare activity is decreasing with stellar age, as expected from other tracers of magnetic activity. In Figure \ref{fig:rossby1} we reproduce the result of \citet{davenport2016}, coloring each point by the stellar mass. The lower panel of Figure \ref{fig:rossby1} shows the $\Psi$ corrected relative flare luminosity, $L_{fl}/L_{bol}$. Since the $\Psi$ correction factor is near 1 for most stars, the change between these panels is modest. Interestingly, a gradient in the $L_{fl}/L_{bol}$ as a function of mass appears for very rapidly rotating stars, particularly those within the ``saturated dynamo'' regime (Ro $< 0.1$). The higher mass (near Solar mass) stars show a larger fraction of their luminosity emitted through flares in this regime. Though our sample of slower rotators is small, this gradient in $L_{fl}/L_{bol}$ seems to disappear for larger Rossby numbers. 

A final complication worth mentioning in the use of the relative flare luminosity metric is due to the varying distances and luminosities of stars in a given magnitude-limited sample. Since detection of flare events (particularly the small-amplitude, lower energy events) depends on the signal to noise of a light curve, the distances to stars can impact the resulting $L_{fl}/L_{Kp}$ measurement. For example, given two identical mass stars with the same underlying flare activity level placed at different distances, the more distant star will have its smaller amplitude flares obscured by photometric noise, and thus a lower $L_{fl}/L_{Kp}$. Similarly, for two stars at a given distance, the lower-mass (fainter) one will have a lower signal-to-noise, and again the flare activity will be under-estimated. As a result, when comparing stars using $L_{fl}/L_{Kp}$ (or $L_{fl}/L_{bol}$), a uniform range of flare event energies must be considered. This presents a major limitation in comparing the flare activity between stars of different masses. We therefore generally recommend $L_{fl}/L_{Kp}$ be used when comparing between stars of similar mass.




%%%%%%%%%%%%%%%%%%%%%%%%%%%%%%
\subsection{Specific Flare Rate}
\label{sec:rate}

Rather than integrating the relative flux from all detected flares within a light curve, as outlined in \S\ref{sec:fracL}, flare activity can be expressed as a specific occurrence frequency. Since observable flares occur with a wide range of energies for a given star, such a frequency should be reported for flares of a given energy {\it or larger} (e.g. 10 flares per year with energies of $10^{32}$ erg or larger). This is equivalent to evaluating the flare frequency distribution (FFD; see Figure \ref{fig:ffd1}) at a given energy. Here we present this metric as $R_{32}$, with the subscript denoting the $\log$ energy that rate is evaluated at, and with units of cumulative number per day.

The specific flare rate has been used in some capacity for many years for comparing flare activity levels between stars \citep[e.g.][]{lme1976}, and was noted by \citet{davenport2016} as an effective metric for comparing flare activity levels between stars at different distances. Since the specific flare rate is generated from the FFD, it can in principle be evaluated at energies {\it not} observed for a given star by fitting and extrapolating a powerlaw function to the FFD. Note this implicitly assumes that the flare rate for a star is governed by a single power law at all energies of interest, which is not always supported by observations. For example, as \citet{davenport2016} highlight for KIC 11551430, a ``break'' in the FFD power law is observed for superflare stars at high event energies. 

Projecting the specific flare rate using the single FFD power law very useful for comparing stars with different observing conditions, such as 1) stars with very different observing durations where one star may not have produced many large flares to compare to, or 2) comparing stars at significantly different distances where small amplitude flares from the fainter object are not detectable. What is required for projecting a flare rate to new energy range, however, is a sufficient number of flares be observed to adequately measure the power law distribution in the FFD.

This specific flare rate metric is also appealing due to its easily understood units, i.e. number of flares per day, and can be useful when considering the impact of flares on other observable properties. For example, \citet{davenport2016b} report for Proxima Cen a specific rate for superflares of $R_{33}=8$ per year that may impact exoplanet habitability, and a rate of $R_{28}=63$ per day for events having amplitudes comparable to an exoplanet transit signal.

%instead of integrating all flares, one can utilize their cumulative flare frequency vs energy distribution (FFD), and evaluate it at a fixed energy. this is appealing because the FFD is the typical figure of merit for considering a flare rate, and has been used in some capacity for many years \citep[e.g.][]{lme1976}, was noted as an option in D16 for effectively comparing flare activity levels for stars at different distances. for each star we can compute this by creating the FFD over whatever range of flares was actually observed, regardless of SpT or distance. note this is a reverse cumulative distribution, meaning the specific rates derived are for flares of a certain energy {\it or larger}. this FFD can be fit with a powerlaw function, and then a flare rate is read off the powerlaw function at a fixed energy for all stars.

\begin{figure}[!t]
\centering
\includegraphics[width=3.5in]{figures/Rossby_rate_color}
\includegraphics[width=3.5in]{figures/color_rot_R35}
\caption{
Specific flare rate ($R_{35}$), evaluated at an energy of log E = 35 erg as a function of the stellar mass and Rossby number (top), and the observed stellar rotation and color (bottom). The sample is the same as in Figure \ref{fig:rossby1}.
}
\label{fig:rossby2}
\end{figure}


In Figure \ref{fig:rossby2} we demonstrate the specific flare rate $R_{35}$ for the same sample of stars shown in Figure \ref{fig:rossby1}. Here we explore both the dependence on mass and Rossby number (top) and the observed color and rotation period (bottom). The energy of $10^{35}$ erg was selected as the average event energy observed in the \citet{davenport2016} flare census. For stars at a given mass or color the specific flare rate declines with increasing rotation period (or Rossby number), consistent with the results from the previous section. The specific flare rate is therefore a simple and effective way to compare flare rates between stars at different distances and under vastly different observing conditions, provided the stars are comparable in color (mass).


However, a gradient in $R_{35}$ in Figure \ref{fig:rossby2} is seen as a function of both mass and color due to the specific flare rates being lower for low-mass stars at a given energy. The specific flare rate shows an unintuitive trend when comparing flare rates between stars of different masses, whereby lower-mass stars appear {\it less} active.
This reproduces the previously known effect that while low-mass stars produce a high rate of observable flares and can produce a large fraction of their luminosity in flares, the actual rate of events with solar-type flare energies is low. 


%This is a good metric for comparing the flare rate of objects within a narrow range of temperature or spectral type, but whose distances and therefore flare energy sensitivity ranges are vastly different. Also good for compressing flare data into a single number that makes use of all flares in FFD.


To overcome this unintuitive behavior, we can correct the specific flare rate for the difference in quiescent stellar luminosity between stars. Rather than pick a single event energy to evaluate all FFD's at, we instead pick a flare energy that scales with the star's luminosity. A flare event with an equivalent duration of $P=1$ second by definition has an energy equal to the quiescent luminosity of the star integrated for 1 second. A sensible energy to choose for evaluating the FFD at therefore would be the star's quiescent luminosity in the \Kepler band. We will denote this adjusted specific flare rate as $R_{1s}$, or literally the flare rate for events with an equivalent duration of 1 second. In Figure \ref{fig:R1s} we show the rotation--color space for the same sample of stars as a function of their $R_{1s}$ specific flare rate. 

Note: the bolometric luminosity would also be a good choice for defining the comparison energy, but as we explored in the development of the $\Psi$ correction factor above, the ratio of $L_{Kp}/L_{bol}$ is typically near 1 for solar-type stars. 



\begin{figure}[!t]
\centering
\includegraphics[width=3.5in]{figures/color_rot_R1s}
\caption{
The Specific flare rate ($R_{1s}$), evaluated at an energy that is equal to an equivalent duration of 1 second
}
\label{fig:R1s}
\end{figure}






%% HERE

%%%%%%%%%%%%%%%%%%%%%%%%%%%%%%
\begin{figure*}[!ht]
\centering
\includegraphics[width=2.25in]{figures/mean_ffd0}
\includegraphics[width=2.25in]{figures/mean_ffd1}
\includegraphics[width=2.25in]{figures/mean_ffd2}\\
\includegraphics[width=2.25in]{figures/mean_ffd3}
\includegraphics[width=2.25in]{figures/mean_ffd4}
\includegraphics[width=2.25in]{figures/mean_ffd5}
\caption{
Average FFD, combining all available quarters of \Kepler data, for stars in the 6 $g-i$ color bins defined in \citet{davenport2016}. Each track is colored as a function of the measured rotation period from \citet{mcquillan2014}. A clear and coherent decrease in the total flare rates is observed as stars slow down (age) within each panel.
}
\label{fig:meanffd}
\end{figure*}


%%%%%%%%%%%%%%%%%%%%%%%%%%%%%%
\subsection{Modeling the Flare Frequency Distribution}
\label{sec:ffd}

Rather than reducing the entire observed sample of flares for a given star down to a single metric, we can instead characterize the flare frequency distribution. the model would produce the mean, or expected, flare frequency distribution for all stars  at any age, mass, flare energy.
this good because FFD is, again, the figure of merit when discussing solar and stellar flare rates. this also good b/c allows us to compare flare energy distributions between stars of different masses, or distances (changes observable energy range).


In Figure \ref{fig:meanffd} we show the FFD for our sample of NNN stars with measured rotation periods and good flare rates, separated into six bins based on their $g-i$ colors. Thus the stars within each bin have roughly similar masses, {\bf ranging from M=1.1 to M=0.3?}. Each FFD shown is actually the mean FFD for a single star, averaged over all available qtrs of \Kepler data.
Each FFD is colored by the rotation period reported by \citet{mcquillan2014}, with increasing rotation periods from red to blue.
a very clear trend in the flare frequency versus rotation periods is seen, with flare rates monotonically deceasing as stars lose angular momentum. Stars of higher mass have higher energy flares recovered, since the star is intrinsically brighter and a given fractional amplitude of flux corresponds to larger luminosity increase. the low-energy cutoff in each FFD is determined by the flare injection tests of D16. A bias is seen where rapid rotators have slightly higher energy cutoffs, due to increased starspot amplitudes that make flare recovery more difficult for the iterative algorithm of D16.
Since colors of stars are in small bins, the energy zeropoints used to transform relative EDs into energies have very small differences. these FFDs show that the flare rate model idea empirically works.



For each star, need to convert the observed units (color, period) in to theoretical units (mass, age)
color into mass via isochrone in D16.  problems here, of course. new version of KIC stats (Stellar-17) provides another way to test this. our masses generally good. very few potential giants in our final sample of rotating stars {\bf CHECK Stellar-17 vs final sample}

age via gyro prescriptions (ruth, etc). problems may exist at older, slower rotators \citep{van-saders2016}, but our sample of flare stars mostly biased towards more active, younger stars.



%will Jen van Saders have new gyro stuff out i can use?
%this is the big plot i want this paper to be all about. even if i make it a few times with diff age metrics




{\bf null detections.... figure those out}




we assume powerlaw decrease in FFD evolution. this not totally absurd, since other activity indicators have similar form, including the relative flare luminosity observed in D16.
We have used the following model to fit the FFD evolution:

\begin{eqnarray}
\label{eqn:model}
& \log \nu = a \log \varepsilon + b\\
a =& a_1 \log t + a_2 m + a_3 \nonumber\\
b =& b_1 \log t + b_2 m + b_3 \nonumber % \log t \, m + b_4
\end{eqnarray}

\noindent
where $\varepsilon$ is the flare energy in erg, $t$ is the stellar age in Myr, $m$ is the stellar mass relative to Solar, and the resulting $\nu$ is the cumulative number of flares per day. we fit this first with a least-sq approach {\bf(implement BSFG-S)}, and then run model with a MCMC {\bf(implement MCMC - add MCMC analysis figures)}. The final parameters in our model are given in Table \ref{tbl:params}.


\begin{deluxetable}{llll}
\tablecolumns{4}
\tablewidth{0pt}
\tabletypesize{\footnotesize}
\tablecaption{Best-fit parameters for the FFD evolution model described in Equation \ref{eqn:model}.
\label{tbl:params}}
%\tablehead{
%	\colhead{($\log t$)}&
%	\colhead{($m$)}&
%	\colhead{($\log t \, m$)}&
%	\colhead{}
%	}
\startdata
$a_1=-0.073$  &  $a_2=0.60$ &    $a_3=0.062$ & $a_4=-0.965$\\
$b_1=2.33$ & $b_2=-18.39$ &  $b_3=-2.337$  &  $b_4=30.69$\\
\enddata
\end{deluxetable}


we tried other parameterization of the model, including a 4th term (cross term), but the BIC wasn't better enough to warrant the increase in the degrees of freedom. Figure \ref{fig:chisq} shows no strong trend in the quality of the model versus either color or period: a reasonable fit all over.

%\begin{figure}[!t]
%\centering
%\includegraphics[width=3.5in]{figures/color_rot_chisq}
%\caption{
%The reduced $\chi^2$ for all stars in our sample compared with our best-fit FFD evolution model. For stars redder than $g-i\approx1$ no significant trend in the model residual is observed, indicating our FFD evolution model is accurately tracing the change in flare rate versus rotation period. For stars bluer stars, however, a modest trend is present, which may indicate a higher order fit is required to trace the evolution for Solar mass stars.
%}
%\label{fig:chisq}
%\end{figure}



now that we've fit all the FFDs together, we can go back and predict the FFDs in our sample to demonstrate the model is working for specific targets. For example, same targets as before are shown in Figure \ref{fig:ffdmodel}, with model fits.


\begin{figure*}[!t]
\centering
\includegraphics[width=2.25in]{figures/6117832_ffd_fit}
\includegraphics[width=2.25in]{figures/7420545_ffd_fit}
\includegraphics[width=2.25in]{figures/10071383_ffd_fit}
\caption{
Flare frequency distributions as shown from Figure \ref{fig:ffd} (black line), but with the resulting flare activity model for each star shown (red line). Note this model is not fit for each star, but instead is developed from our entire sample.}
\label{fig:ffdmodel}
\end{figure*}






%%%%%%%%%%%%%%%%%%%%%%%%%%%%
\section{Exploring the Flare Rate Evolution}
\label{sec:model}

test the model out in detail. diff masses, diff ages....

\begin{figure}[!t]
\centering
\includegraphics[width=3.5in]{figures/eqnFFD_mass1}
\includegraphics[width=3.5in]{figures/eqnFFD_mass2}
\caption{
Our best-fit flare evolution model showing the predicted FFD evaluated at four age bins for a $0.5 M_\odot$ (top) and $1.0 M_\odot$ star (bottom). While the change in FFD slopes as a function of age is negligible, the difference as a function of mass is significant. Note: These FFDs are drawn at the same energy ranges for ease of comparison, but do not correspond to the specific flare energies detected at these masses.
}
\label{fig:model}
\end{figure}



even make pretty grids, which capture our expectation that M dwarfs flare a lot and for a long time.
\begin{figure}[!t]
\centering
\includegraphics[width=3.5in]{figures/eqnFFD_grid35}
\includegraphics[width=3.5in]{figures/eqnFFD_grid_R1s}
\caption{
Our best-fit FFD evolution model evaluated over a grid of masses and ages, showing the cumulative flare rate $R_{35}$ for a fixed energy of $10^35$ erg (top), and at an equivalent duration of 1 second, i.e. $R_{1s} = $ 1 sec $\times L_{quies}$ (bottom). The latter clearly shows that low-mass stars produce more flares relative to their quiescent luminosity, while all stars show a decrease in their specific flare rate over time.
}
\label{fig:grid}
\end{figure}









%%%%%%%%%%%%%%%%%%%
%\section{Flares versus H$\alpha$ Emission}
%
%made this in Columbia - is there an age correlation we can make? (akin to the LX/Lbol vs age plot - is there an Halpha vs age?)
%
%if we add clusters, Stephanie has data for Praesepe
%
%\begin{figure}[!t]
%\centering
%\includegraphics[width=3.5in]{figures/flare_vs_lhalbol.png}
%\caption{
%H$\alpha$ vs flare energy
%}
%\label{fig:halpha}
%\end{figure}
%



%%%%%%%%%%%%%%%%%%%
%\section{Flares versus Galactic Height}
%Cumulative distribution of flare stars as a function of galactic latitude (or height if I'm willing to make the calculation)
%Height should be easy enough, just use distance I'm assuming from isochrone fits
%
%Similar finding to \citet{kowalski2009} and \citet{hilton2010} for flare stars being more confined to the galactic plane than ``active'' stars (measured from H$\alpha$.) This is consistent with model of flares being sensitive to rotation {\it earlier}, this flare activity dropping sooner in a stars life (lower scale height in Galaxy), and thus a break from saturation at smaller Rossby Number




%%%%%%%%%%%%%%%%%%%%%%%%%%%%%%
%\section{Kinematic ages?}
%use Gaia - good?

%%%%%%%%%%%%%%%%%%%%%%%%%%%%%%
\section{Discussion}
\label{sec:discussion}

this is the first time a direct connection between stellar flare energy and other traditional magnetic activity indicators has been determined, tying the occurrence of flares on small size scales to the total chromospheric activity of the star.


flares in clusters not analyzed here, but basically no flares observed in original \Kepler cluster sample

could flares be an important piece of the lower stellar atmosphere? models of ellerman bombs, etc \citep[e.g.][]{hansteen2017}. \Kepler cannot see flares small enough in energy. However, if the nano/micro-flare rate evolves coherantly with the large/super-flare rates observed here, this atmospheric support mechanism must change.





%%%%%%%%%%%%%%%%%
\acknowledgments
JRAD is supported by an NSF Astronomy and Astrophysics Postdoctoral Fellowship under award AST-1501418.

JRAD acknowledges support from the DIRAC Institute in the Department of Astronomy at the University of Washington. The DIRAC Institute is supported through generous gifts from the Charles and Lisa Simonyi Fund for Arts and Sciences, and the Washington Research Foundation


Kepler was competitively selected as the tenth Discovery mission. Funding for this mission is provided by NASA�s Science Mission Directorate.



\bibliography{/Users/james/Dropbox/references}


\end{document}
